\documentclass[figure]{standalone}
\usepackage{amsmath}
\usepackage{amsthm}
\usepackage{amsfonts}
\usepackage{braket}
\usepackage{quantikz}
\usetikzlibrary{external}
\usepackage{graphicx}
\usepackage{hyperref}
% \usepackage{tikz}
\usetikzlibrary{patterns}
\usepackage{calc}
\usepackage[ruled,vlined]{algorithm2e}
\usepackage[nopatch]{microtype}





% \usepackage{xparse}
\tikzset{
    gateX/.style={
        append after command={
            \pgfextra {
                \node at ([shift={(-0.01,-0.01)}] \tikzlastnode.north east) {?};
            }
        }
    },
    gateXbottom/.style={
        append after command={
            \pgfextra {
                \node at ([shift={(-0.01,-0.01)}] \tikzlastnode.north east) {?};
                % \node[anchor=north] at (\tikzlastnode.south ) {#1};
            }
        }
    }
}
\newcommand{\Oplus}{\ensuremath{\vcenter{\hbox{\scalebox{1.5}{$\oplus$}}}}}

\DeclareExpandableDocumentCommand{\gateX}{O{}m}{%
    |[gateX,#1]| {#2} \qw
}

\DeclareExpandableDocumentCommand{\gateXbottom}{O{}m}{%
    |[gateXbottom={#2},#1]| {#2} \qw
}


\usepackage{tikz}
\usetikzlibrary{backgrounds}
\usetikzlibrary{arrows}
\usetikzlibrary{shapes,shapes.geometric,shapes.misc}

% this style is applied by default to any tikzpicture included via \tikzfig
\tikzstyle{tikzfig}=[baseline=-0.25em,scale=0.5]

% these are dummy properties used by TikZiT, but ignored by LaTex
\pgfkeys{/tikz/tikzit fill/.initial=0}
\pgfkeys{/tikz/tikzit draw/.initial=0}
\pgfkeys{/tikz/tikzit shape/.initial=0}
\pgfkeys{/tikz/tikzit category/.initial=0}

% standard layers used in .tikz files
\pgfdeclarelayer{edgelayer}
\pgfdeclarelayer{nodelayer}
\pgfsetlayers{background,edgelayer,nodelayer,main}

% style for blank nodes
\tikzstyle{none}=[inner sep=0mm]

% include a .tikz file
\newcommand{\tikzfig}[1]{%
{\tikzstyle{every picture}=[tikzfig]
\IfFileExists{#1.tikz}
  {\input{#1.tikz}}
  {%
    \IfFileExists{./figures/#1.tikz}
      {\input{./figures/#1.tikz}}
      {\tikz[baseline=-0.5em]{\node[draw=red,font=\color{red},fill=red!10!white] {\textit{#1}};}}%
  }}%
}

% the same as \tikzfig, but in a {center} environment
\newcommand{\ctikzfig}[1]{%
\begin{center}\rm
  \tikzfig{#1}
\end{center}}

% fix strange self-loops, which are PGF/TikZ default
\tikzstyle{every loop}=[]

% TiKZ style file generated by TikZiT. You may edit this file manually,
% but some things (e.g. comments) may be overwritten. To be readable in
% TikZiT, the only non-comment lines must be of the form:
% \tikzstyle{NAME}=[PROPERTY LIST]

% TiKZ style file generated by TikZiT. You may edit this file manually,
% but some things (e.g. comments) may be overwritten. To be readable in
% TikZiT, the only non-comment lines must be of the form:
% \tikzstyle{NAME}=[PROPERTY LIST]

% Node styles
% TiKZ style file generated by TikZiT. You may edit this file manually,
% but some things (e.g. comments) may be overwritten. To be readable in
% TikZiT, the only non-comment lines must be of the form:
% \tikzstyle{NAME}=[PROPERTY LIST]

% Node styles
\tikzstyle{rectangle black}=[fill={rgb,255: red,64; green,64; blue,64}, draw=black, shape=rectangle]
\tikzstyle{new style 0}=[fill=black, draw=black, shape=circle]
\tikzstyle{CNOT}=[fill=none, draw=black, shape=circle, tikzit draw=black, new atom]

% Edge styles
\tikzstyle{dashed edge}=[-, dashed, dash pattern=on 4mm off 2mm]
\tikzstyle{thick edge}=[-, fill={rgb,255: red,64; green,64; blue,64}, thick]














\begin{document}
\begin{quantikz}[
    font=\footnotesize,
    row sep={0.6cm,between origins},
    column sep=0.5cm,
    classical gap=0.1cm,
    wire types={q,q,n,q,n,n,q,q,q,q,q,q}
]
    \lstick{$a_2$}&\octrl{1}&\octrl{1}&\octrl{1} &\octrl{1}&\ctrl{1}&\ctrl{1} &\ctrl{1}&\ctrl{1}&\rstick{$a_2$}\\
    \lstick{$a_1$}&\octrl{2}&\octrl{2}&\ctrl{2} &\ctrl{2}&\octrl{2}&\octrl{2} &\ctrl{2}&\ctrl{2}&\rstick{$a_1$}\\
    &&& &&& &&&\\
    \lstick{$a_0$}&\octrl{8}&\ctrl{8}&\octrl{8} &\ctrl{8}&\octrl{8}&\ctrl{8} &\octrl{8}&\ctrl{8}&\rstick{$a_0$}\\
    &&& &&& &&&\\
    &&& &&& &&&\\
    \lstick[6]{$d$}&\gateX{\Oplus}&\gateX{\Oplus}&\gateX{\Oplus} &\gateX{\Oplus}&\gateX{\Oplus}&\gateX{\Oplus} &\gateX{\Oplus}&\gateX{\Oplus}&\\
    &\gateX{\Oplus}&\gateX{\Oplus}&\gateX{\Oplus} &\gateX{\Oplus}&\gateX{\Oplus}&\gateX{\Oplus} &\gateX{\Oplus}&\gateX{\Oplus}&\\
    &\gateX{\Oplus}&\gateX{\Oplus}&\gateX{\Oplus} &\gateX{\Oplus}&\gateX{\Oplus}&\gateX{\Oplus} &\gateX{\Oplus}&\gateX{\Oplus}&\\
    &\gateX{\Oplus}&\gateX{\Oplus}&\gateX{\Oplus} &\gateX{\Oplus}&\gateX{\Oplus}&\gateX{\Oplus} &\gateX{\Oplus}&\gateX{\Oplus}&\\
    &\gateX{\Oplus}&\gateX{\Oplus}&\gateX{\Oplus} &\gateX{\Oplus}&\gateX{\Oplus}&\gateX{\Oplus} &\gateX{\Oplus}&\gateX{\Oplus}&\\
    % &\gateX{\Oplus}&\gateX{\Oplus}&\gateX{\Oplus} &\gateX{\Oplus}&\gateX{\Oplus}&\gateX{\Oplus} &\gateX{\Oplus}&\gateX{\Oplus}&\\
    &\gateXbottom{\Oplus}&\gateXbottom{\Oplus}&\gateXbottom{\Oplus}&\gateXbottom{\Oplus}&\gateXbottom{\Oplus}&\gateXbottom{\Oplus}&\gateXbottom{\Oplus}&\gateXbottom{\Oplus}&\\
    \setwiretype{n}&\push{T_0}&\push{T_1}&\push{T_2}&\push{T_3}&\push{T_4}&\push{T_5}&\push{T_6}&\push{T_7}&\\
\end{quantikz}=\begin{quantikz}[
    font=\footnotesize,
    row sep={0.6cm,between origins},
    column sep=0.2cm,
    classical gap=0.1cm,
    wire types={q,q,n,q,n,n,q,q,q,q,q,q}
]
    \lstick{$a_2$}&\octrl{1}&& &&\octrl{1}& &\octrl{1}&& &&\octrl{1}&\push{\cdots} &\ctrl{1}&& &&\ctrl{1}& &\ctrl{1}&& &&\ctrl{1}&\\
    \lstick{$a_1$}&\octrl{1}&& &&\octrl{1}& &\octrl{1}&& &&\octrl{1}&\push{\cdots} &\ctrl{1}&& &&\ctrl{1}& &\ctrl{1}&& &&\ctrl{1}&\\
    &&\ctrl{1}\setwiretype{q}& &\ctrl{1}&&\setwiretype{n} &&\ctrl{1}\setwiretype{q}& &\ctrl{1}&&\setwiretype{n}\push{\cdots} &&\ctrl{1}\setwiretype{q}& &\ctrl{1}&&\setwiretype{n} &&\ctrl{1}\setwiretype{q}& &\ctrl{1}&&\setwiretype{n}\\
    \lstick{$a_0$}&&\octrl{1}& &\octrl{1}&& &&\ctrl{1}& &\ctrl{1}&&\push{\cdots} &&\octrl{1}& &\octrl{1}&& &&\ctrl{1}& &\ctrl{1}&&\\
    &&&\ctrl{7}\setwiretype{q} &&\setwiretype{n}& &&&\ctrl{7}\setwiretype{q} &&\setwiretype{n}&\push{\cdots} &&&\ctrl{7}\setwiretype{q} &&\setwiretype{n}& &&&\ctrl{7}\setwiretype{q} &&\setwiretype{n}&\\
    &&& &&& &&& &&& &&& &&& &&& &&&\\
    &&&\gateX{\Oplus} &&& &&&\gateX{\Oplus} &&&\push{\cdots} &&& \gateX{\Oplus}&&& &&&\gateX{\Oplus} &&&\\
    &&&\gateX{\Oplus} &&& &&&\gateX{\Oplus} &&&\push{\cdots} &&& \gateX{\Oplus}&&& &&&\gateX{\Oplus} &&&\\
    &&&\gateX{\Oplus} &&& &&&\gateX{\Oplus} &&&\push{\cdots} &&& \gateX{\Oplus}&&& &&&\gateX{\Oplus} &&&\\
    &&&\gateX{\Oplus} &&& &&&\gateX{\Oplus} &&&\push{\cdots} &&& \gateX{\Oplus}&&& &&&\gateX{\Oplus} &&&\\
    &&&\gateX{\Oplus} &&& &&&\gateX{\Oplus} &&&\push{\cdots} &&& \gateX{\Oplus}&&& &&&\gateX{\Oplus} &&&\\
    &&&\gateX{\Oplus} &&& &&&\gateX{\Oplus} &&&\push{\cdots} &&& \gateX{\Oplus}&&& &&&\gateX{\Oplus} &&&\\
    \setwiretype{n}&&&\push{T_0} &&& &&&\push{T_1} &&&\push{\cdots} &&& \push{T_6}&&& &&&\push{T_7} &&&
\end{quantikz}
\end{document}

