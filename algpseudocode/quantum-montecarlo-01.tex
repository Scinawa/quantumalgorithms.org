\documentclass{article}
\usepackage[utf8]{inputenc}

\usepackage{algorithm}
\usepackage{algpseudocode}

\usepackage{braket}
\usepackage{amsmath}
\usepackage{amssymb}
\usepackage{caption}


\newcommand{\norm}[1]{\left\lVert#1\right\rVert}

\makeatletter
\renewcommand{\fnum@algorithm}{\fname@algorithm}
\makeatother

\begin{document}
\pagestyle{empty}

\begin{algorithm}[ht]
	\caption{Monte Carlo with bounded output}
	\begin{algorithmic}[1]

		\Require  A quantum algoritm $A$ on $n$ qubits such that $0\leq \nu(A) \leq 1$, integer $t$, real $\delta>0$.
		\Ensure An estimate of $\mathbb{E}[\nu(A)]$.
		\vspace{10pt}
		\Statex
	%\State If necessary, modify $A$ such that it makes no measurement until the end of the algorithm; operates on initial input state $\ket{0^n}$; and its final measurement is a measurement of the last $k\leq n$ of these qubits in the computational basis.-->

		\State Let $W$ be the unitary operator on $k+1$ qubits defined by
		       \begin{equation}
		       W\ket{x}\ket{0} =\ket{x}\big(\sqrt{1-\phi(x)}\ket{0}+\sqrt{\phi(x)}\ket{1}
		       \end{equation}
		       where each computational basis state $x \in \{0,1\}^k$ is associated with a real number $\phi(x) \in [0,1]$, such that $\phi(x)$ is the value output by $A$ when measurement $x$ is recieved.

		 \State Repeat the following step $O(\log(1/\delta)$ times and output the median of the results:
		 Apply $t$ iterations of amplitude estimation, setting $\ket{\psi} = (I \otimes W)(A \otimes I)\ket{0^{n+1}}$, $P = I \otimes |1\rangle\langle 1|$.


	\end{algorithmic}
\end{algorithm}

\end{document}
